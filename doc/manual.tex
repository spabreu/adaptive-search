\documentclass{article}

\usepackage{fancyheadings}

% General environments

\newenvironment{Indentation}%
   {\begin{list}{}{}%
      \item }%
   {\end{list}}

\newenvironment{Code}%
   {\begin{Indentation}\begin{tt}}%
   {\end{tt}\end{Indentation}}

\newenvironment{VCode}%
   {\begin{Indentation}\begin{verbatim}}%
   {\end{verbatim}\end{Indentation}}

\begin{document}

\newlength{\saveparskip}
\setlength{\saveparskip}{\parskip}

\pagestyle{empty}
\setlength{\parskip}{0pt}
~

\vspace{4cm}
{\huge\bf Adaptive Search}

\vspace{3mm}

\rule[2mm]{\linewidth}{2mm}

\begin{flushright}

{\Large
A Library to Solve CSPs\\
Edition 1.1, for Adaptive Search version \input{version_no}\\

\today
}

\end{flushright}


\vspace{10cm}

{\large\bf by Daniel Diaz, Philippe Codognet and Salvador Abreu}

\rule[2mm]{\linewidth}{1mm}

\newpage
~

\vspace{10cm}

\setlength{\parskip}{\saveparskip}

Copyright (C) 2002-2010 Daniel Diaz, Philippe Codognet and Salvador Abreu



Permission is granted to make and distribute verbatim copies of this manual
provided the copyright notice and this permission notice are preserved on all
copies.

Permission is granted to copy and distribute modified versions of this manual
under the conditions for verbatim copying, provided that the entire resulting
derived work is distributed under the terms of a permission notice identical
to this one.

Permission is granted to copy and distribute translations of this manual into
another language, under the above conditions for modified versions, except
that this permission notice may be stated in a translation approved by the
Free Software Foundation, 59 Temple Place - Suite 330, Boston, MA 02111, USA.


\newpage
\pagestyle{fancy}

\setcounter{page}{1}




\section{Introduction}

The Adaptive Search library provides a set of functions to solve CSPs by a
local search method. For more information consult \cite{saga01} The current
release only works for problems that can be stated as permutation
problems. More precisely, all $n$ variables have a same domain $x_1~..~x_n$
and are subject to an implicit \textit{all-different} constraint. Several
problems fall into this category and some examples are provided with the
library.

\section{Installation}

Please refer to the file called \texttt{INSTALL} located in the src subdirectory.

\section{The Adaptive Search API}

\subsection{Overall usage}

The typical use of the API is as follows:

\begin{itemize}

\item Initialize a structure with the input data needed by the solver. This
  includes problem data (e.g. size, domain,...) together with parameters to
  tune the solver (e.g. tebu tenure,...).

\item Define a set of functions needed by the solver (e.g. to compute the
  cost of a configuration). Some functions are optional meaning that the
  solver performs an implicit treatment in the absence of such a
  function. Most of the times, providing an optional function speeds up the
  execution.

\item Call the solver.

\item Exploit the data provided by the solver (the solution, various counters
  giving information about the resolutions).

\end{itemize}


To use the API a C file should include the header file
\texttt{ad\_solver.h}:

\begin{Code}
\begin{verbatim}
#include "ad_solver.h"
\end{verbatim}
\end{Code}

Obviously the C compiler must be invoked with the adequate option to ensure
the header file can be found by the preprocessor.

At link time, the library called \texttt{libad\_solver.a} must be
passed. Here also, some options might have to be passed to the C compiler to
allow the linker to locate the library.

If both the include file and the library are in the same directory as the
user C file (for instance \texttt{problem.c}), then the following Unix command
line (using gcc) suffices:

\begin{Code}
\begin{verbatim}
gcc -o problem problem.c libad_solver.a
\end{verbatim}
\end{Code}

If the include file is in \texttt{/usr/adaptive/include} and the library in
 \texttt{/usr/adaptive/lib}, a possible invocation could be:

\begin{Code}
\begin{verbatim}
gcc -I/usr/adaptive/include -L/usr/adaptive/lib -o problem problem.c -lad_solver
\end{verbatim}
\end{Code}

A structure (C type  \texttt{AdData}) is used to communicate with the solver. Fields in this structure can be decomposed in: input or output data (or input-output). Input parameters are given to the solver and should be initialized before calling the solver. Output parameters are provided by the solver.

Please look at the header file for more information about the fields in the
\texttt{AdData} type. We here detail the most important parameters.




\subsection{Input parameters}

The following input variables control the basic data and have to be
initialized before calling the resolution function.

\begin{itemize}

\item \texttt{int size}: size of the problem (number of variables).

\item \texttt{int *sol}: the array of variables. It is an output parameter
  but it can also be used to pass the initial configuration if \texttt{int
    do\_not\_init} is set.

\item \texttt{int do\_not\_init}: if set to true (a value != 0) the initial
 configuration used is the one present in \texttt{sol} (else a random configuration
 is computed).

\item \texttt{int base\_value}: base offset for the domain of each variable
 (each vaiable can then take a value in
 $\texttt{base\_value}~..~\texttt{base\_value} + \texttt{size}-1$. 

\item \texttt{int *actual\_value}: if not \texttt{NULL} it contains the array
  of values (domain) variables can take. If \texttt{base\_value} is given, it
  is added to each value of \texttt{actual\_value} to form the domain.

\item \texttt{int break\_nl}: when the solver displays a solution a new
 line is inserted every \texttt{break\_nl} values (0 if no break is
 wanted). This makes it possible to display matrix in a more readable form.

\item \texttt{int debug}: debug level (0: none, 1: trace, 2: interactive).
 This requires the library is compiled with debugging support (see INSTALL).

\item \texttt{char *log\_file}: name of the log file (or \texttt{NULL}
 if none).  This requires the library is compiled with log file support (see
 INSTALL).

\end{itemize}

The following input parameters make it possible to tune the solver and should
be initialized before calling the resolution function.

\begin{itemize}		

\item \texttt{int exhaustive}: if true the solver always evaluate (the
 cost of) all possible swaps to chose the best swap. If false a projection of
 the error on each variable is used to first select the ``worst'' variable
 (\cite{saga01} for more information).

\item \texttt{int first\_best}: when looking for the next configuration, the
  solver stops as soon as a better move is found (instead of continuing to
  find the best move).

\item \texttt{int prob\_select\_loc\_min}: this is a percentage to force a
 local minimum (i.e. when the 2 selected variables to swap are the same)
 instead of staying on a plateau (a swap involves 2 different variables but
 the overall cost will remain the same). If a value $>$ 100 is given, this
 option is not used.

\item \texttt{int freeze\_loc\_min}: number of swaps a variable is frozen
 when a local minimum is encountered (i.e. the 2 variables to swap are the
 same).

\item \texttt{int freeze\_swap}: number of swaps the 2 variables that have
 been selected (and thus swapped) to improve the solution are frozen.

\item \texttt{int reset\_limit}: number of frozen varables before a reset
 is triggered.

\item \texttt{int nb\_var\_to\_reset}: number of variables to randomly
 reset.

\item \texttt{int restart\_limit}: maximum number of iterations before
 restarting from scratch (give a big number to avoid a restart).

\item \texttt{int restart\_max}: maximal number of restarts to perform before
  giving up. To avoid a too long computation the parameters \texttt{int
    restart\_limit} and \texttt{int restart\_max} can be defined.

\item \texttt{int reinit\_after\_if\_swap}: see the defintion of the user 
 function \texttt{Cost\_If\_Swap()} for more information.

\end{itemize}

\subsection{Output parameters}

In addition to the array containing the solution, the solver maintains
counters that can be consulted by the user to obtain some information about
the resolution.

\begin{itemize}

\item \texttt{int *sol}: the current configuration. When the solver
  teminates, it normally contains a solution. If the solver has finished
  because it reached the maximum number of iterations and restarts, the
  \texttt{sol} array contains a pseudo-solution (an aprroximation of the
  solution).

\item \texttt{int total\_cost}: cost of the current configuration (0 means a
  solution).

\item \texttt{int nb\_restart}: number of restart performed.

\item \texttt{int nb\_iter}, \texttt{int nb\_iter\_tot}: number of iterations performed in the current
  pass and across restarts.

\item \texttt{int nb\_swap}, \texttt{int nb\_swap\_tot}: number of swaps performed.

\item \texttt{int nb\_same\_var}, \texttt{int nb\_same\_var\_tot}: number of
  variables with (the same) highest cost.

\item \texttt{int nb\_reset}, texttt{int nb\_reset\_tot}: number of reset swaps performed.

\item \texttt{int nb\_local\_min}, \texttt{int nb\_local\_min\_tot}: number of local minimum encountered.

\end{itemize}


\subsection{Miscellaneous parameters}

The following variables are not used by the solver. They simply convey values
for the user. It is particularly useful for multithreading. It also contains
some information related to the default \texttt{main()} function.

\begin{itemize}

\item \texttt{int param}: the parameter handled by the default \texttt{main()} function.

\item \texttt{int seed}: the seed set by a command-line option of the default \texttt{main()}
  function (or -1 if any). 

\item \texttt{int reset\_percent}: -1 or the \% of variables to reset defined
  by a command-line option of the default \texttt{main()} function.  If it is
  -1, the \texttt{Init\_Parameters()} function should either set it to a
  percentage or directly set the \texttt{nb\_var\_reset} parameter.
 
\item \texttt{int data32[4]}: some values to store 32-bits user information.

\item \texttt{int data64[42}: some values to store 64-bits user information.

\end{itemize}


\subsection{Functions}

Here is the set of functions provided by the library:

\begin{itemize}

\item \texttt{int Ad\_Solve(AdData *p\_ad)}: this function invokes the Adaptive solver
 to find a solution to the problem. This function calls in turn user
 functions (e.g. to compute the cost of a solution or to project this cost on
 a given variable). This function returns the \texttt{total\_cost} at
 then end of the resolution (i.e. 0 if a solution has been found).

\item \texttt{void Ad\_Display(int *t, AdData *p\_ad, unsigned *mark)}: this function displays
 an array \texttt{t} (generally \texttt{sol}) and also displays a 'X' for
 marked variables (if \texttt{mark != NULL}). This function is generally only 
 used by the solver.

\end{itemize}


\subsection{User functions}

The function \texttt{Ad\_Solve()} calls some user functions to guide its
resolution. Some functions are MANDATORY while others are OPTIONAL. Here is
the set of user functions:

\begin{itemize}

\item \texttt{int Cost\_Of\_Solution(int should\_be\_recorded)}: [MANDATORY]
  this function returns the cost of the current solution (the user code
  should keep a pointer to \texttt{sol} it needed). The argument
  \texttt{should\_be\_recorded} is passed by the solver, if true the solver
  will continue with this cost (so maybe the user code needs to register some
  information), if false the solver simply wants to know the cost of a
  possible move (but without electing it).

\item \texttt{int Cost\_On\_Variable(int i)}: [OPTIONAL] this function
 returns the projection of the current cost on the \texttt{i}\textit{th}
 variable (from 0 to \texttt{size}-1). If not present then the
 resolution must be exhausitive (see \texttt{exhausitive}).

\item \texttt{int Cost\_If\_Swap(int current\_cost, int i, int j)}: [OPTIONAL]
  this function evaluates the cost of a swap (the swap is not performed and
  should not be performed by the function). Passed arguments are the cost of
  the solution, the indexes \texttt{i} and\texttt{j} of the 2 candidates for
  a swap.. If this function is not present a default function is used which:
 \begin{itemize}

   \item performs the swap, 

   \item calls \texttt{Cost\_Of\_Solution()},

   \item undoes the swap,

   \item if the variable \texttt{int reinit\_after\_if\_swap} is true
         then \texttt{Cost\_Of\_Solution()} is also called another time. This
         is useful if \texttt{Cost\_Of\_Solution()} updates some global
         information to ensure this information is reset.
 \end{itemize}

\item \texttt{void Executed\_Swap(int i, int j)}: [OPTIONAL] this function is
 called to inform the user code a swap has been done. This is useful if the
 user code maintains some global information.

\item \texttt{int Next\_I(int i)}: [OPTIONAL] this function is called in case
 of an exhaustive search (see \texttt{exhaustive}). It is used to
 enumerate the first variable. This functions receives the current \texttt{i}
 (initially it is -1) and returns the next value (or something $>$
 \texttt{size} at the end). In case this function is not defined,
 \texttt{i} takes the values $0~..~\texttt{size} - 1$.

\item \texttt{int Next\_J(int i, int j)}: [OPTIONAL] this function is called
 in case of an exhaustive search (see \texttt{exhaustive}). It is used to
 enumerate the second variable. This functions receives the current
 \texttt{i} and the current \texttt{j} (for each new \texttt{i} it is -1) and
 returns the next value for \texttt{j} (or something $>$ \texttt{size} at
 the end). In case this function is not defined, \texttt{j} takes the values
 $\texttt{i}+1~..~\texttt{size}-1$ for each new $\texttt{i}$.

\item \texttt{void Display\_Solution(AdData *p\_ad)}: [OPTIONAL] this
  function is called to display a solution (stored inside \texttt{sol}). This
  allows the user to customize the output (useful if modelisation of the
  problem needs a decoding to be understood). The default version simply
  displays the values in \texttt{sol}.

\end{itemize}

\section{Other utility functions}

To use this functions the user C code should include the file
\texttt{tools.h}.

\begin{itemize}

\item \texttt{long Real\_Time(void)}: returns the real elapsed time since the
  start of the process (wall time).

\item \texttt{long User\_Time(void)}: returns the user time since the start
 of the process.

\item \texttt{unsigned Randomize\_Seed(unsigned seed)}: intializes the
 random generator with a given \texttt{seed}.

\item \texttt{unsigned Randomize(void)}: randomly initilizes the random
 generator.

\item \texttt{unsigned Random(unsigned n)}: returns a random integer $>= 0$ 
 and  $< \texttt{n}$.

\item \texttt{void Random\_Permut(int *vec, int size, const int
    *actual\_value, int base\_value)}: initializes the \texttt{size} elements
  of the vector \texttt{vec} with a random permutation. If
  \texttt{actual\_value} is \texttt{NULL}, values are taken in
  $base\_value~..~\texttt{size}-1+base\_value$. If \texttt{actual\_value} is
  given, values are take from this array (each element of the array is added
  to \texttt{base\_value} to form an element of the permutation).

\item \texttt{int Random\_Permut\_Check(int *vec, int size, const int
    *actual\_value, int base\_value)}: checks if the values of \texttt{vec}
  forms a valid permutation (returns true or false).

\item \texttt{void Random\_Permut\_(int *vec, int size, const int
    *actual\_value, int base\_value)}: repairs the permutation stored in
  \texttt{vec} so that it now contains a valid permutation (trying to keep
  untouched as much as possible good values).


\end{itemize}


\section{Using the default \texttt{main()} function}

The user is obviously free to write his own \texttt{main()} function. In
order to have a same command-line options for all bechmarks a default
\texttt{main()} is included in the library (it is then used if no user
\texttt{main()} is found at link-time). The default function act as follows:

\begin{itemize}

\item it parses the command-line to retrieve tuning options, the running mode
  (number of executions,...), and the parameter if it is expected (e.g. the
  chessboard size in the queens). NB: if a parameter is expected the variable
  \texttt{param\_needed} must be declared and initialized to 1 in the user
  code.  Each tuning option can be set via a command-line option and the
  corresponding variable (see input variables) is set. The only exception is
  for for \texttt{nb\_var\_reset} which can be specified indirectly as a
  percentage (instead of an absolute value) inside the variable
  \texttt{reset\_percent}.

\item it invokes a user function \texttt{void Init\_Parameters(AdData *p\_ad)} which
 must initialises all input variables (e.g. \texttt{size},
 allocate \texttt{sol},...). This function should only initialises tuning variable
 that are not set via command-line options. In this case the value of the
 corresponding variable is -1. 

\item it invokes the user defined \texttt{void Solve(AdData *p\_ad)} function
  (which in turn should invoke the Adaptive solver \texttt{Ad\_Solve()}.

\item it displays the result or a summary of the counters (in benchmark mode).

\end{itemize}


In addition to the variables described above, the following parameters
are available when using the default \texttt{main()}:

\begin{itemize}

\item \texttt{int param}: the parameter (if \texttt{param\_needed} is true).

\item \texttt{int reset\_percent}: -1 or the \% of variables to reset defined
 by a command-line option.  If it is -1, the \texttt{Initializations()}
 function should either set it to a percentage or directly set
 \texttt{nb\_var\_reset}.

\item \texttt{int seed}: -1 or the seed defined by a command-line option. The
 \texttt{main()} function initialises the random number generator in both
 cases so the \texttt{Initializations()} does not need to do it.

\end{itemize}

The default default \texttt{main()} function needs an additional function to
check the validity of a solution. The user must then provide a function
\texttt{int Check\_Solution(AdData *p\_ad)} which returns 1 if the solution
passed in \texttt{p\_ad} is valid. A Simple definition for this function
could be to simply test if the cost of the solution is 0 (not very precise as
verification).

Please look at the examples for more details.



\begin{thebibliography}{99}

\bibitem{saga01}
\newblock P. Codognet and D. Diaz.
\newblock Yet Another Local Search Method for Constraint Solving.
\newblock In {\em Proc. SAGA01}, 1st International Symposim on
Stochastic Algorithms : Foundations and Applications,
\newblock LNCS 2246, Springer Verlag 2001

\end{thebibliography}

\end{document}
